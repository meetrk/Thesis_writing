\chapter{Approach}
\label{chap:approach}

In this chapter, We describe how we have designed and organized our approach to 


\section{Model Architecture}
\label{sec:model_architecture}

\begin{itemize}
    \item Encoder - Relation Graph Convolutional Network (R-GCN) 
    \item Decoder - Knowledge Graph Embedding (KGE) - Dismult \cite{yang2014embedding}
    \item Autoencoder Approach for link prediction\cite{schlichtkrull2017modelingrelationaldatagraph}
\end{itemize}

\section{Regularisation Techniques}
\label{sec:regularisation_techniques}

\itemize
    \item add mathematical description of eaach technique
    \item Negative Sampling - negative edges are sampled using simple random sampling on number of entities. 
    \item Edge Dropout - dropping edges during message passing and loss is computed on all train edges. \cite{rong2019dropedge}
    \item Label Smoothing - adding noise to one-hot encoded labels.\cite{szegedy2016rethinking}
\enditemize


\section{Uncertainty Estimation}
\label{sec:uncertainty_estimation}

\subsection{Monte Carlo Dropout}
\label{subsec:monte_carlo_dropout}

\begin{itemize}
    \item having dropout after every layer in R-GCN\cite{gal2016dropoutbayesianapproximationrepresenting}
    \item during inference, perform T stochastic forward passes with dropout enabled
\end{itemize}


\subsection{Deep Ensembles}
\label{subsec:deep_ensembles}

\begin{itemize}
    \item training M models with different random initialisations\cite{lakshminarayanan2017simplescalablepredictiveuncertainty}
    \item during inference, perform prediction with all M models and aggregate results
\end{itemize}


\section{Uncertainty Calibration}
\label{sec:uncertainty_calibration}

\subsection{Parameteric Approaches}
\label{subsec:parametric_approaches}

\subsubsection{Parameterised Temperature Scaling (PTS)}
\label{subsubsec:parameterised_temperature_scaling}
using neural network for input based temperature scaling. \cite{tomani2022parameterizedtemperaturescalingboosting}

\subsection{Non-Parametric Approaches}
\label{subsec:non_parametric_approaches}

\subsubsection{Scalar Temperature Scaling}
\label{subsubsec:scalar_temperature_scaling}
Using a scalar value for scaling the probabilities of the neural network\cite{guo2017calibrationmodernneuralnetworks}

\subsubsection{Isotonic Regression}
\label{subsubsec:isotonic_regression}
Using regression to fit the probability according to probabilities\cite{guo2017calibrationmodernneuralnetworks}

