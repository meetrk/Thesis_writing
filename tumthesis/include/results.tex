\chapter{Results}
\label{chap:results}

\section{Experiment 1: Regularization Impact}
\label{sec:experiment_1_regularization_impact}
This section presents the results of Experiment 1, which investigates the impact of different regularization techniques on the performance and uncertainty calibration of RGCN model. We compare the baseline model with models employing edge dropout, label smoothing, and a combination of both techniques.


\subsection{Link Prediction Performance}
\label{subsec:link_prediction_performance}

Table \ref{tab:link_prediction_performance} summarizes the performance of various models on the WN18RR and FB15k-237 datasets using standard link prediction metrics: Mean Reciprocal Rank (MRR) and Hits@K (K=1,3,10).

\begin{itemize}
    \item \textbf{Baseline}: The baseline RGCN model is a vanilla version where we have not applied edge dropout or label smoothing. We can see that the baseline model achieves an MRR of 0.396 in WN18RR dataset and 0.222 in FB15k-237 dataset. Hits@1 for both datasets are 0.376 and 0.131 respectively. These results serves as a reference point for evaluating the impace of regularization techniques.
    \item \textbf{Edge Dropout}: Introducing Edge dropout with rates of 0.1 and 0.2 improves the MRR to 0.404 and 0.405 in the WN18RR and to 0.254 and 0.255 in FB15k-237, respectively. The Hits@1 and Hits@3 metrics also show improvement with edge dropout, particularly at the 0.2 rate, which achieves the highest scores across all metrics for both datasets.
    \item \textbf{Label Smoothing}: Applying label smoothing with values of $\omega = 0.1$ and $\omega = 0.2$ yields MRRs of 0.407 and 0.396 for WN18RR, respectively. The Hits@1 metric also shows improvement, particularly with $\omega = 0.1$, reaching 0.386. While, for FB15K-237, the performance was totally opposite as MRR dropped to 0.221 and 0.222 respectively and Hits@1 also dropped to 0.132 for both values of $\omega$. 
    \item \textbf{Combined Techniques}: For WN18RR, The best performance is observed when both edge dropout (0.2) and label smoothing (0.1) are combinely applied, achieving highest MRR of 0.4386 and Hits@1 of 0.40. This combination yields highest scoress across all metrics, indicating a synergistic effect of the two regularization methods. However, for FB15k-237, the combined techniques was better than baseline and label smoothing but not better than edge dropout alone. 
\end{itemize}


\begin{table}[htbp]
        \centering
        \includegraphics[width=\textwidth]{figures/experiment1/exp1table}
        \caption{Link Prediction Performance on WN18RR and FB15k-237 Datasets}
        \label{tab:link_prediction_performance}
\end{table}




\subsection{Uncertainty Metrics Analysis}
\label{subsec:uncertainty_metrics_analysis}

Figure \ref{tab:uncertainty_metrics_analysis} illustrates the impact of different regularization techniques on uncertainty estimation metrics, including Expected Calibration Error (ECE) and Reliability Diagram. Key observations include:
\begin{itemize}
    \item \textbf{Basline} shows the highest ECE among all models, indicating worse calibration compared to regularized models. Both of the models are under confident in their predictions as we can see in reliability diagram \ref{tab:uncertainty_metrics_analysis} where the red line is above the diagonal line for both datasets. For WN18RR, the ECE is 0.318, while for FB15K-237, the ECE is 0.3003. We can also observe that most of the predictions are concentrated in the lower confidence bins near 0. This shows that the RGCN model are not well calibrated and needs calibration techniques to make them reliable in real world applications.

    \item \textbf{Edge Dropout} The ECE score for edge dropout models are not significantly different from the baseline, with values of 0.299 and 0.301 for WN18RR and for FB15K-237 0.271 and 0.254 for edge dropout rates of 0.1 and 0.2, respectively. It is interesting to note that nature of the reliability diagram for edge dropout models is similar to the baseline, with predictions still concentrated in the lower confidence bins.
    

    \item \textbf{Label Smoothing} Label smoothing models show a signicant change in the models behavior as the ECE scores dropped and the reliability diagram shows that the model changed from squeeze the predictions in the lower confidence bins to more spread out across the confidence spectrum. Best performance is observed with $\omega = 0.2$ for both datasets, achieving ECE scores dropping by 19\% for WN18RR and by 36\% for FB15K-237 compared to the baseline. This can be observed in the reliability diagram where the orange line is much closer to the diagonal line compared to other models. And lowest and highest confidence bin are near 0.1 and 0.9 for $\omega = 0.2$ which shows that label smoothing is working as expected.
    
    \item \textbf{Combined Techniques} The combination of edge dropout and label smoothing yields the best calibration performance, with ECE scores of 0.238 for WN18RR and 0.158 for FB15K-237. This is a signicant improvement over the baseline and individual regularization techniques. In the reliability diagram, the yellow line represents the combined techniques and is closest to the diagonal line and the confidence bins are more evenly distributed across the spectrum as compared to the baseline and other regularization techniques. 
\end{itemize}


\begin{table}[htbp]
        \centering
        \begin{minipage}{0.48\textwidth}
                \centering
                \includegraphics[width=\textwidth]{figures/experiment1/exp1graph2}
        \end{minipage}
        \hfill
        \begin{minipage}{0.48\textwidth}
                \centering
                \includegraphics[width=\textwidth]{figures/experiment1/exp1graph3}
        \end{minipage}
        \caption{Uncertainty Metrics Analysis on WN18RR and FB15K-237 Datasets with Different Regularization Techniques. Here Edge Dropout is denoted as ED and Label Smoothing as LS.}
        \label{tab:uncertainty_metrics_analysis}
\end{table}


\section{Experiment 2: Uncertainty Estimation Methods}
\label{sec:experiment_2_uncertainty_estimation_methods}

This section presents the results of Experiment 2, which evaluates monte carlo dropout (MC Dropout) and deep ensembles as uncertainty estimation methods applied to the best performing regularized RGCN model from Experiment 1 - For WN18RR, Baseline + ED + LS and For FB15K-237, Baseline + ED. The performance and uncertainty estimation metrics are compared against the baseline regularized model without uncertainty estimation.

\subsection{Link Prediction Performance}
\label{subsec:link_prediction_performance_exp2}
Table \ref{tab:link_prediction_performance_exp2} summarizes the link prediction performance of the uncertainty estimation methods on the WN18RR and FB15k-237 datasets. Key observations include:

\begin{itemize}
        \item \textbf{Impact of } 
        \item \textbf{Basline} 
        \item \textbf{Basline} 
        \item \textbf{Basline} 
\end{itemize} 

\begin{table}[htbp]
        \centering
        \includegraphics[width=1\textwidth]{figures/experiment2/table}
        \caption{Link prediction performance of Uncertainty Estimation Methods on WN18RR and FB15k-237 Datasets}
        \label{tab:link_prediction_performance_exp2}
\end{table}


\subsection{Reliability Diagrams}
\label{subsec:reliability_diagrams}

Here are the description of the reliability diagrams for WN18RR and FB15K-237 datasets with different uncertainty estimation methods:



\begin{table}[htbp]
        \centering
        \begin{minipage}{0.48\textwidth}
                \centering
                \includegraphics[width=\textwidth]{figures/experiment2/graph_wrr}
        \end{minipage}
        \hfill
        \begin{minipage}{0.48\textwidth}
                \centering
                \includegraphics[width=\textwidth]{figures/experiment2/graph_fb15}
        \end{minipage}
        \caption{Uncertainty Metrics Analysis on WN18RR and FB15K-237 Datasets with Different Uncertainty Estimation Methods. Here Edge Dropout is denoted as ED and Label Smoothing as LS.}
        \label{tab:uncertainty_metrics_analysis_exp2}
\end{table}


\subsection{Computational Overhead (OPTIONAL)}
\label{subsec:computational_overhead}

\section{Experiment 3: Calibration Methods}
\label{sec:experiment_3_calibration_methods}

\begin{table}[htbp]
        \centering
        \begin{minipage}{0.48\textwidth}
                \centering
                \includegraphics[width=\textwidth]{figures/experiment3/graph_rea_wn18}
        \end{minipage}
        \hfill
        \begin{minipage}{0.48\textwidth}
                \centering
                \includegraphics[width=\textwidth]{figures/experiment3/graph_rea_fb15k}
        \end{minipage}
        \caption{Link prediction performance of Uncertainty Estimation Methods on WN18RR and FB15K-237}
        \label{tab:reliability_diagram_wn18rr_exp3}
\end{table}




\begin{table}[htbp]
        \centering
        \begin{minipage}{0.48\textwidth}
                \centering
                \includegraphics[width=\textwidth]{figures/experiment3/graph_wrr}
        \end{minipage}
        \hfill
        \begin{minipage}{0.48\textwidth}
                \centering
                \includegraphics[width=\textwidth]{figures/experiment3/graph_fb15k}
        \end{minipage}
        \caption{Uncertainty Metrics Analysis on WN18RR and FB15K-237 Datasets with Different Uncertainty Estimation Methods. Here Edge Dropout is denoted as ED and Label Smoothing as LS.}
        \label{tab:compare_metrics_exp3}
\end{table}


\subsection{Impact on ECE and ACE}
\label{subsec:impact_on_ece_and_ace}

\subsection{Qualitative Case Studies (OPTIONAL)}
\label{subsec:qualitative_case_studies}

\section{Experiment 4: Hybrid Approaches}
\label{sec:experiment_4_hybrid_approaches}



\begin{table}[htbp]
        \centering
        \includegraphics[width=1\textwidth]{figures/experiment4/table}
        \caption{Link prediction performance of Uncertainty Estimation Methods on FB15K-237}
        \label{tab:ece_table_exp4}
\end{table}


\subsection{Best Combined Configurations}
\label{subsec:best_combined_configurations}


\begin{table}[htbp]
        \centering
        \begin{minipage}{0.48\textwidth}
                \centering
                \includegraphics[width=\textwidth]{figures/experiment4/figure_wn_mc}
        \end{minipage}
        \hfill
        \begin{minipage}{0.48\textwidth}
                \centering
                \includegraphics[width=\textwidth]{figures/experiment4/figure_wn_en}
        \end{minipage}
        \caption{Reliability Curves on WN18RR Dataset with Different Calibration Methods applied to Monte Carlo Dropout and Deep Ensemble Uncertainty Estimation Methods.}
        \label{tab:compare_metrics_wn_exp4}
\end{table}


\begin{table}[htbp]
        \centering
        \begin{minipage}{0.48\textwidth}
                \centering
                \includegraphics[width=\textwidth]{figures/experiment4/figure_fb_mc}
        \end{minipage}
        \hfill
        \begin{minipage}{0.48\textwidth}
                \centering
                \includegraphics[width=\textwidth]{figures/experiment4/figure_fb_en}
        \end{minipage}
        \caption{Reliability Curves on FB15K-237 Dataset with Different Calibration Methods applied to Monte Carlo Dropout and Deep Ensemble Uncertainty Estimation Methods.}
        \label{tab:compare_metrics_fb_exp4}
\end{table}
